\title{Homework 1}
\author{Zachary Moring}
\documentclass{article}

\usepackage{amsmath,amssymb}
\usepackage{boxproof}

\newcommand{\Hide}[1]{}             % use \Hide{bla bla} to hide ``bla bla''
\newcommand{\code}[1]{\texttt{#1}}  % use \code{...} to produce ASCII chars
\newcommand{\Intro}[1]{{#1}{\textrm{i}}}
\newcommand{\Elim}[1]{{#1}{\textrm{e}}}

\begin{document}
\maketitle

\noindent\textbf{Exercise 1.2.1(h).} Prove the validity of
\[
    p \vdash \left( p \to q \right) \to q.
\]

\vspace{0.2cm}
\hrule{}
\vspace{0.2cm}

\begin{proofbox}
    \: p                            \= \textrm{premise} \\
    \[
        \: p \to q                  \= \textrm{assume} \\
        \: q                        \= \textrm{\Elim{\to} 1~2}
    \]
    \: \left( p \to q \right) \to q \= \textrm{\Intro{\to} 2~:~3}
\end{proofbox}

\newpage{}

\noindent\textbf{Exercise 1.2.1(i).} Prove the validity of
\[
    \left( p \to r \right) \wedge \left(q \to r \right) \vdash p \wedge q \to r.
\]

\vspace{0.2cm}
\hrule{}
\vspace{0.2cm}

\begin{proofbox}
    \: \left( p \to r \right) \wedge \left( q \to r \right) \= \textrm{premise} \\
    \[
        \: p \wedge q                                       \= \textrm{assume} \\
        \: p                                                \= \textrm{\Elim{\wedge}1 2} \\
        \: p \to r                                          \= \textrm{\Elim{\wedge}1 1} \\
        \: r                                                \= \textrm{\Elim{\to} 3~4} \\
    \]
    \: p \wedge q \to r                                     \= \textrm{\Intro{\to} 2~:~5}
\end{proofbox}

\newpage{}

\noindent\textbf{Exercise 1.2.1(j).} Prove the validity of
\[
    q \to r \vdash \left( p \to q \right) \to \left( p \to r \right).
\]

\vspace{0.2cm}
\hrule{}
\vspace{0.2cm}

\begin{proofbox}
    \: q \to r                \= \textrm{premise} \\
    \[
        \: p \to q            \= \textrm{assume} \\
        \[
            \: p              \= \textrm{assume} \\
            \: q              \= \textrm{\Elim{\to} 2~3} \\
            \: r              \= \textrm{\Elim{\to} 1~4} \\
        \]
        \: p \to r            \= \textrm{\Intro{\to} 3~:~5} \\
    \]
    \: \left( p \to q \right) \= \textrm{\Intro{\to} 2~:~6} \\
\end{proofbox}

\newpage{}

\noindent\textbf{Exercise 1.4.2(g).} Provide the truth table for
\[
    \left( \left( p \to q \right) \to p \right) \to p.
\]

\vspace{0.2cm}
\hrule{}
\vspace{0.2cm}

This statement is a tautology:

\begin{centering}
\begin{table}[h!]
\begin{tabular}{ccccc}
\textbf{$p$} & \textbf{$q$}           & \textbf{$p \to q$} & \textbf{$\left( p \to q \right) \to p$} & \textbf{$\left( \left( p \to q \right) \to p \right) \to p$} \\ \hline
T            & \multicolumn{1}{c|}{T} & T                  & T                                       & T                                                            \\
T            & \multicolumn{1}{c|}{F} & F                  & T                                       & T                                                            \\
F            & \multicolumn{1}{c|}{T} & T                  & F                                       & T                                                            \\
F            & \multicolumn{1}{c|}{F} & T                  & F                                       & T                                                            \\
\end{tabular}
\end{table}
\end{centering}

\newpage{}

\noindent\textbf{Exercise 1.4.2(h).} Provide the truth table for
\[
    \left( \left( p \vee q \right) \to r \right) \to \left( \left( p \to r \right) \vee \left( q \to r \right) \right).
\]

\vspace{0.2cm}
\hrule{}
\vspace{0.2cm}

This statement is a tautology:

\begin{table}[h!]
\begin{tabular}{ccc|cc}
$p$ & $q$ & $r$ & $p \vee q$ & $\left( p \vee q \right) \to r$ \\ \hline
T   & T   & T   & T          & T \\
T   & T   & F   & T          & F \\
T   & F   & T   & T          & T \\
T   & F   & F   & T          & F \\
F   & T   & T   & T          & T \\
F   & T   & F   & T          & F \\
F   & F   & T   & F          & T \\
F   & F   & F   & F          & T \\
\end{tabular}
\end{table}

\begin{table}[h!]
\begin{tabular}{ccc|ccc}
$p$ & $q$ & $r$ & $p \to r$ & $q \to r$ & $\left( p \to r\right ) \vee \left( q \to r \right)$ \\ \hline
T   & T   & T   & T & T & T \\
T   & T   & F   & F & F & F \\
T   & F   & T   & T & T & T \\
T   & F   & F   & F & T & T \\
F   & T   & T   & T & T & T \\
F   & T   & F   & T & F & T \\
F   & F   & T   & T & T & T \\
F   & F   & F   & T & T & T \\
\end{tabular}
\end{table}

\begin{table}[h!]
\begin{tabular}{ccc|c}
$p$ & $q$ & $r$ & $\left( \left(p \vee q \right) \to r \right) \to \left( \left( p \to r \right) \vee \left(q \to r \right) \right)$ \\ \hline
T   & T   & T   & T \\
T   & T   & F   & T \\
T   & F   & T   & T \\
T   & F   & F   & T \\
F   & T   & T   & T \\
F   & T   & F   & T \\
F   & F   & T   & T \\
F   & F   & F   & T \\
\end{tabular}
\end{table}

\newpage{}

\noindent\textbf{Exercise 1.4.2(i).} Prove the truth table for
\[
    \left( p \to q \right) \to \left( \lnot p \to \lnot q \right).
\]

\vspace{0.2cm}
\hrule{}
\vspace{0.2cm}

\begin{table}[h!]
\begin{tabular}{cc|ccccc}
$p$ & $q$ & $\lnot p$ & $\lnot q$ & $p \to q$ & $\lnot p \to \lnot q$ & $\left( p \to q \right) \to \left( \lnot p \to \lnot q \right)$ \\ \hline
T   & T   & F         & F         & T         & T                     & T \\
T   & F   & F         & T         & F         & T                     & T \\
F   & T   & T         & F         & T         & F                     & F \\
F   & F   & T         & T         & T         & T                     & T \\
\end{tabular}
\end{table}

\newpage{}

\noindent\textbf{Problem 1.5.3(b).} Show that if $C \subseteq \left\{
\lnot, \wedge, \vee, \to, \perp \right\}$ is adequate for propositional
logic, then $\lnot \in C$ or $\perp \in C$.

\vspace{0.2cm}
\hrule{}
\vspace{0.2cm}

Let $C \subseteq \left\{ \lnot, \wedge, \vee, \to, \perp \right\}$ be a subset
and assume that $\lnot \not\in C$ and $\perp \not\in C$; then, $C \subseteq
\left\{ \wedge, \vee, \to \right\}$.

To demonstrate that $C$ is not adequate for propositional logic, we only need
to demonstrate that there exists at least one statement of propositional logic
which cannot be expressed only through the connectives in $C$.

Consider the propositional logic statement $\psi = \lnot p \wedge \lnot q$.
Assume that $\varphi$ is a formula formed only with the connectives in $C$
which is equivalent to $\psi$. Then $\varphi$ must produce \verb|F| when
$p$ and $q$ are both \verb|T|.

However, $\varphi$ can only be made of some combination of the connectives
$\wedge$, $\vee$, and $\to$, all of which assign \verb|T| when the input atoms
are both \verb|T|. Thus, no combination of strictly these symbols can produce
\verb|F| when $p$ and $q$ are both true; to do so, we need some mechanism to
invert the truth value, and we have none since $\lnot \not\in C$. Further,
since $\perp \not\in C$, we cannot even use a contradiction to reach the output
we desire.

Thus, there is no formula $\varphi$ over the connectives in $C$ which is
equivalent to $\psi$, and thus $C$ is not adequate for propositional logic.

Therefore, by the contrapositive, if $C$ is adequate for propositional logic,
then $\lnot \in C$ or $\perp \in C$.

\newpage{}

\noindent\textbf{Problem 1.5.3(c).} Is $\left\{ \leftrightarrow, \lnot \right\}$
adequate? Prove your answer.

\vspace{0.2cm}
\hrule{}
\vspace{0.2cm}

Consider the combinations of $\leftrightarrow$ and $\lnot$. First, observe that
$p \leftrightarrow q$ has exactly two \verb|T| outputs and two \verb|F| outputs.
Likewise, $p \leftarrow \lnot q$, etc., all also have exactly two \verb|T|
outputs and two \verb|F| outputs. Applying $\lnot$ to any statement composed
of $\leftrightarrow$ connections will flip all \verb|T| to \verb|F| and vice
versa, producing in turn a statement with two \verb|T| outcomes and two
\verb|F| outcomes. Thus, a combination of $\lnot$ and $\leftrightarrow$ on
two input atoms can only produce statements with two \verb|T| and two
\verb|F| outcomes.

Now consider $p \wedge q$, a statement with one \verb|T| output and three
\verb|F| outputs. By our observation on the properties of $\lnot$ and
$\leftrightarrow$, no combination of $\lnot$ and $\leftrightarrow$ on two
input atoms can produce a valuation with three \verb|F| outputs and one
\verb|T| output.

Thus, no combination of $\lnot$ and $\leftrightarrow$ on two input atoms
can produce the same truth table as $p \wedge q$.

Therefore, there exists at least one statement of formal logic which cannot
be modeled using only $\leftrightarrow$ and $\lnot$.

Therefore, $\left\{ \leftrightarrow, \lnot \right\}$ is not adequate for
formal logic.

\end{document}
