\title{Homework 3}
\author{Zachary Moring}
\documentclass{article}

\usepackage{amsmath,amssymb}
\usepackage{boxproof}
\usepackage{hyperref}
\usepackage{tikz}
\usepackage{qtree}

\newcommand{\Hide}[1]{}
\newcommand{\code}[1]{\texttt{#1}}
\newcommand{\Intro}[1]{{#1}{\textrm{i}}}
\newcommand{\Elim}[1]{{#1}{\textrm{e}}}
\newcommand{\Premise}{\textrm{premise}}
\newcommand{\Assume}{\textrm{assume}}

\newcommand*{\Break}{\vspace{0.2cm}\hrule{}\vspace{0.2cm}}

\begin{document} \maketitle

Code for this assignment is at: \url{https://github.com/zpm-bu/cs511-formal-methods/blob/assignments/lean/Homework/hw3.lean}

\newpage{}

\noindent\textbf{Exercise 1, Part 1.} Write a natural-deduction proof of the
following WFF: $\varphi_1 \overset{\Delta}{=} \lnot (p \wedge q \wedge r) \to
(\lnot p \vee \lnot q \vee \lnot r)$, de Morgan's law (4).

\Break{}

\begin{proofbox}
  \[
    \label{a1}\: \lnot \left( p \wedge q \wedge r \right) \= \Assume \\
    \[
      \label{a2}\: \lnot \left( \lnot p \vee \lnot q \vee \lnot r
      \right) \= \Assume \\
      \[
        \label{a3}\: \lnot r \= \Assume \\
        \label{a4}\: \lnot p \vee \lnot q \vee \lnot r \=
        \Intro{\vee}_2~\ref{a3} \\
        \label{a5}\: \perp \= \Intro{\perp}~\ref{a4},\ref{a2} \\
      \]
      \label{a6}\: \lnot \lnot r \= \Intro{\lnot}~\ref{a3}:\ref{a5} \\
      \label{a7}\: r \= \Elim{\lnot\lnot}~\ref{a6} \\
      \[
        \label{a8}\: \lnot p \vee \lnot q \= \Assume \\
        \label{a9}\: \lnot p \vee \lnot q \vee \lnot r \=
        \Intro{\vee}_1~\ref{a8} \\
        \label{a10}\: \perp \= \Intro{\perp}~\ref{a2},\ref{a9} \\
      \]
      \label{a11}\: \lnot (\lnot p \vee \lnot q) \=
      \Intro{\lnot}~\ref{a8}:\ref{a10} \\
      \[
        \label{a12}\: \lnot q \= \Assume \\
        \label{a13}\: \lnot p \vee \lnot q \= \Intro{\vee}_2~\ref{a12} \\
        \label{a14}\: \perp \= \Intro{\perp}~\ref{a2},\ref{a12} \\
      \]
      \label{a15}\: \lnot\lnot q \= \Intro{\lnot}~\ref{a12}:\ref{a14} \\
      \label{a16}\: q \= \Elim{\lnot\lnot}~\ref{a15} \\
      \[
        \label{a17}\: \lnot p \= \Assume \\
        \label{a18}\: \lnot p \vee \lnot q \= \Intro{\vee}_1~\ref{a17} \\
        \label{a19}\: \perp \= \Intro{\perp}~\ref{a2},\ref{a18} \\
      \]
      \label{a20}\: \lnot\lnot p \= \Intro{\lnot}~\ref{a17}:\ref{a19} \\
      \label{a21}\: p \= \Elim{\lnot\lnot}~\ref{a19} \\
      \label{a22}\: p \wedge q \= \Intro{\wedge}~\ref{a16},\ref{a21} \\
      \label{a23}\: p \wedge q \wedge r \= \Intro{\wedge}~\ref{a22},\ref{a7} \\
      \label{a24}\: \perp \= \Intro{\perp}~\ref{a1},\ref{a23}
    \]
    \label{a25}\: \lnot\lnot(\lnot p \vee \lnot q \vee \lnot r) \=
    \Intro{\lnot}~\ref{a2}:\ref{a24} \\
    \label{a26}\: \lnot p \vee \lnot q \vee \lnot r \=
    \Elim{\lnot\lnot}~\ref{a25} \\
  \]
  \: \lnot \left( p \wedge q \wedge r \right) \to (\lnot p \vee \lnot q \vee \lnot r)
  \= \Intro{\to}~\ref{a1}:\ref{a26}
\end{proofbox}

\newpage{}

\noindent\textbf{Exercise 1, Part 2.} Write a natural-deduction proof of the
most generalized form of de Morgan's law (4),
\[
  \varphi_2 \overset{\Delta}{=} \lnot(p_1 \wedge p_2 \wedge \dots \wedge p_n)
  \to (\lnot p_1 \vee \lnot p_2 \vee \dots \vee \lnot p_n).
\]

\Break{}

\begin{proofbox}
\[
  \label{b1}\: \lnot\left( p_1 \wedge \dots \wedge p_n \right) \= \Premise \\
  \[
    \label{b2}\: \lnot (\lnot p_1 \vee \dots \vee \lnot p_n) \= \Assume \\
    \[
      \label{b3}\: \lnot p_n \= \Assume \\
      \label{b4}\: \left( \lnot p_1 \vee \dots \right) \vee \lnot p_n \= \Intro{\vee}_2~\ref{b3} \\
      \label{b5}\: \perp \= \Intro{\perp}~\ref{b2},\ref{b4} \\
    \]
    \label{b6}\: \lnot \lnot p_n \= \Intro{\lnot}~\ref{b3}:\ref{b5} \\
    \label{b7}\: p_n \= \Elim{\lnot\lnot}~\ref{b6} \\
    \[
      \label{b8}\: \lnot p_1 \vee \dots \vee \lnot p_{n-1} \= \Assume \\
      \label{b9}\: \lnot p_1 \vee \dots \vee \lnot p_{n-1} \vee \lnot p_n \= \Intro{\vee}_1~\ref{b8} \\
      \label{b10}\: \perp \= \Intro{\perp}~\ref{b2},\ref{b9} \\
    \]
    \label{b11}\: \lnot (\lnot p_1 \vee \dots \vee \lnot p_{n-1}) \= \Intro{\lnot}~\ref{b8}:\ref{b10} \\
    \[
      \label{b12}\: \lnot p_{n-1} \= \Assume \\
      \label{b13}\: \left( \lnot p_1 \vee \dots \vee \lnot p_{n-2} \right) \vee p_{n-1} \= \Intro{\vee}_2~\ref{b12} \\
      \label{b14}\: \perp \= \Intro{\perp}~\ref{b11},\ref{b13} \\
    \]
    \label{b15}\: \lnot\lnot p_{n-1} \= \Intro{\lnot}~\ref{b12}:\ref{b14} \\
    \label{b16}\: p_{n-1} \= \Elim{\lnot\lnot}~\ref{b15} \\
    \: \vdots \= \vdots \\
    \label{b18}\: p_1 \wedge \dots \wedge p_n \= \Intro{\wedge} \\
    \label{b19}\: \perp \= \Intro{\perp}~\ref{b1},\ref{b18} \\
  \]
  \label{b20}\: \lnot \lnot (\lnot p_1 \vee \dots \vee \lnot p_n) \= \Intro{\lnot}~\textrm{box} \\
  \label{b21}\: \lnot p_1 \vee \dots \vee \lnot p_n \= \Elim{\lnot\lnot}~\ref{b20} \\
\]
  \: \lnot \left( p_1 \wedge \dots \wedge p_n \right) \to (\lnot p_1 \vee \dots \vee \lnot p_n)
  \= \Intro{\to}~\ref{b1}:\ref{b21}
\end{proofbox}

\newpage{}

\noindent\textbf{Exercise 1, Part 3.} Show that there is a natural-deduction
proof of the generalized de Morgan's law (4) whose length (defined as the
number of lines in the proof) is $O(n)$.

\Break{}

We use the formal proof in Exercise 1, Part 2 as the template for constructing
a proof in $O(n)$ steps.

Note that, as shown in steps 3-11, we can perform a sequence of conclusions in
9 steps which strips the last $p_i$ from the $\vee$ sequence and reduces the
length of the $\vee$ sequence by 1 proposition. We conclude $p_i$ as well as
$p_1 \vee \dots \vee p_{i-1}$, and repeat. Each of these operations is in
a constant number of lines, repeated $n$ times, once for each $p$. Our total
steps is $O(8 \times n) = O(n)$.

Then, we need $n$ additional steps to stitch the various $p_i$ together into
the conclusion $p_1 \wedge \dots \wedge p_n$, contradicting the premise.

This proof is overall $O(n)$ steps. Thus, by construction, we build a proof of
$O(n)$ steps.

\newpage{}

\noindent\textbf{Exercise 1, Part 4.} Compare the complexity of
a natural-deduction proof of $\varphi_2$ and the complexity of a truth-table
verification of $\varphi_2$.

\Break{}

The natural-deduction proof of $\varphi_2$ can be done in a linear number of
steps $O(n)$, as demonstrated by construction. The truth-table approach, on the
other hand, requires a truth table with $2^n$ rows for each combination of the
$n$ binary-value inputs. As a result, this is a problem for which the
natural-deduction approach is significantly easier.

\newpage{}

\noindent\textbf{Exercise 2, Part 1.} Use the tableaux method to show the
validity of the following more general version of de Morgan's law (4):
\[
  \varphi_1 \overset{\Delta}{=} \lnot (p \wedge q \wedge r) \to (\lnot p \vee \lnot q \vee \lnot r)
\]

\Break{}

\Tree [.{$\lnot( \lnot (p \wedge q \wedge r) \to (\lnot p \vee \lnot q \vee \lnot r) )$}
  [.{$\lnot(p \wedge q \wedge r)$}\\{$\lnot(\lnot p \vee \lnot q \vee \lnot r)$}
    [.{$\lnot(p \wedge q \wedge r)$}\\{$\lnot(\lnot p \vee \lnot q)$}\\{$\lnot \lnot r$}
      [.{$\lnot(p \wedge q \wedge r)$}\\{$\lnot\lnot p$}\\{$\lnot \lnot q$}\\{$\lnot \lnot r$}
        [.{$\lnot (p \wedge q \wedge r)$}\\{$p$}\\{$\lnot \lnot q$}\\{$\lnot\lnot r$}
          [.{$\lnot (p \wedge q \wedge r)$}\\{$p$}\\{$q$}\\{$\lnot\lnot r$}
            [.{$\lnot (p \wedge q \wedge r)$}\\{$p$}\\{$q$}\\{$r$}
              [.{$\lnot(p \wedge q \wedge r)$}\\{$p$}\\{$q$}\\{$r$}
                [.{$\lnot (p \wedge q)$}
                  [.{$\lnot p$}\\{\times} ]
                  [.{$\lnot q$}\\{\times} ]
                ]
                [.{$\lnot r$}\\{\times} ]
              ]
            ]
          ]
        ]
      ]
    ]
  ]
]

\newpage{}

\noindent\textbf{Exercise 2, Part 2.} Use the tableaux method to show the
validity of de Morgan's law (4) in general:
\[
  \varphi_2 \overset{\Delta}{=} \lnot (p_1 \wedge \dots \wedge p_2) \to
    (\lnot p_1 \vee \dots \vee \lnot p_n)
\]

\Break{}

\Tree [.{$\lnot(\lnot(p_1 \wedge \dots \wedge p_n) \to (\lnot p_1 \vee \dots \vee \lnot p_n))$}
  [.{$\lnot (p_1 \wedge \dots \wedge p_n)$}\\{$\lnot(\lnot p_1 \vee \dots \vee \lnot p_n)$}
    [.{\textbf{Repeat $n$ times:}}\\{$\lnot (p_1 \wedge \dots \wedge p_n)$}\\{$\lnot(\lnot p_1 \vee \dots \vee \lnot p_{n-1})$}\\{$\lnot \lnot p_n$}
      [.{$\lnot (p_1 \wedge \dots \wedge p_n)$}\\{$\lnot\lnot p_n$}\\{$\vdots$}\\{$\lnot\lnot p_1$}
        [.{\textbf{Simplify $n$ times:}}\\{$\lnot (p_1 \wedge \dots \wedge p_n)$}\\{$ p_n$}\\{$\vdots$}\\{$ p_1$}
          [.{$\lnot(p_1 \wedge \dots \wedge p_{n-1})$}
            [.{$\lnot (p_1 \wedge \dots \wedge p_{n-2})$}\\{$\vdots$}\\{$\times$} ]
            [.{$\lnot p_{n-1}$}\\{$\times$} ]
          ]
          [.{$\lnot p_n$}\\{$\times$} ]
        ]
      ]
    ]
  ]
]

\newpage{}

\noindent\textbf{Exercise 2, Part 3.} Compute the precise size of the tableau
in Part 2 above, as a function of $n$.

\Break{}

We can enumerate the nodes as follows:

The initial node is the negation of the implication, 1.

The next node is the one below it, splitting the implication into two parts, for
2 nodes.

Then, we need to add $n$ additional nodes to get the $\lnot \lnot p_i$
statements, so $n + 2$ nodes so far.

We then need to cancel each of the $\lnot \lnot$ nodes, which agains requires
$n$ more nodes. That brings the total to $2n + 2$.

That leaves us in a state with $\lnot (p_1 \wedge \dots \wedge p_n)$ and
the additional $n$ statements $p_i$.

Each simplification from here produces 2 nodes, a left and right branch, which
immediately close by contradicting one of the $p_i$ statements. That is an
additional $2n$ nodes, for a total of $4n + 2$.

Lastly, note that the final cancellation is actually $\lnot(p_1 \wedge p_2)$,
which splits into two branches which immediately close. Thus, there actually
is no node with $\lnot p_1$ on the right, so we counted 2 too many nodes.

The final node count is $4n$.

\newpage{}

\noindent\textbf{Exercise 2, Problem 4.} Compare the comlexity of the tableau
proof for $\varphi_2$ in part 2 above with the complexity of the natural
deduction proof of $\varphi_2$ and the truth-table verification of $\varphi_2$.

\Break{}

As established, both the natural-deduction proof and the tableau proof have
a linear size complexity (number of steps and nubmer of nodes, respectively).
This seems to line up with the intuitive expectation that these two methods
are roughly equivalent.

In both cases, they are more efficient than the truth table, which requires
$2^n$ rows.

\end{document}
