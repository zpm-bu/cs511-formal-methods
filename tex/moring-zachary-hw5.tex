\documentclass{article}
\title{Homework 4}
\author{Zachary Moring}

\usepackage{amsmath,amssymb}
\usepackage{boxproof}
\usepackage{hyperref}
\usepackage{tikz}

\newcommand{\Hide}[1]{}
\newcommand{\code}[1]{\texttt{#1}}
\newcommand{\Intro}[1]{{#1}{\textrm{i}}}
\newcommand{\Elim}[1]{{#1}{\textrm{e}}}
\newcommand{\Premise}{\textrm{premise}}
\newcommand{\Let}{\textrm{let}}
\newcommand{\Assume}{\textrm{assume}}
\newcommand{\Break}{\vspace{0.2cm}\hrule{}\vspace{0.2cm}}
\newcommand{\tIf}{\text{if }}
\newcommand{\all}{\forall}

\begin{document} \maketitle

Code for this assignment is at:
\url{https://github.com/zpm-bu/cs511-formal-methods/blob/assignments/lean/Homework/hw5.lean}

\vspace{1in}

\noindent\textbf{Exercise 1a.} Prove the validity of the following sequent:
\[
  (y \approx 0) \wedge (y \approx x) \vdash 0 \approx x
\]

\Break{}

\begin{proofbox}
  \: (y \approx 0) \wedge (y \approx x) \= \Premise \\
  \: y \approx 0 \= \Elim{\wedge}_1~1 \\
  \: y \approx x \= \Elim{\wedge}_2~1 \\
  \: 0 \approx x \= \Elim{\approx}~2,3~(\varphi := y \approx x; t_1
  := y; t_2 := 0)\\
\end{proofbox}

\newpage{}

\noindent\textbf{Exercise 1b.} Prove the validity of the following sequent:
\[
  t_1 = t_2 \vdash (t + t_2) = (t + t_1)
\]

\Break{}

I'm definitely not sure about this one.  Can we introduce an
arbitrary symbol like this?

\begin{proofbox}
  \: t_1 \approx t_2 \= \Premise \\
  t + t_2 \: \Let \\
  \: t + t_2 \approx t + t_2 \= \Intro{\approx}~2 \\
  \: t + t_1 \approx t + t_2 \= \Elim{\approx}~1,3~(\varphi := t +
  t_2 \approx t + t_2; t_1 := t_2; t_2: t_1) \\
\end{proofbox}

\newpage{}

\noindent\textbf{Exercise 2a.} Prove the validity of the following sequent:
\[
  (\exists x (S \to Q(x)) \vdash S \to \exists x Q(x)
\]

\Break{}

\begin{proofbox}
  \: \exists x (S \to Q(x)) \= \Premise \\
  \[
    \: S \= \Assume \\
    \[
      x_0 \: \Let \\
      \: S \to Q(x_0) \= \Assume \\
      \: Q(x_0) \= \Elim{\to}~2,4 \\
      \: \exists x (Q(x)) \= \Intro{\exists}~5 \\
    \]
    \: \exists x (Q(x)) \= \Elim{\exists}~1,3:6 \\
  \]
  \: S \to \exists x (Q(x)) \= \Intro{\to}~2:7 \\
\end{proofbox}

\newpage{}

\noindent\textbf{Exercise 2d.} Prove the validity of the following sequent:
\[
  \all x (P(x)) \to S \vdash \exists x (P(x) \to S)
\]

\Break{}

\begin{proofbox}
  \: \all x (P(x)) \to S \= \Premise \\
  \[
    \: \lnot \exists x (P(x) \to S) \= \Assume \\
    \[
      x_0 \: \Let \\
      \[
        \: \lnot P(x_0) \= \Assume \\
        \[
          \: P(x_0) \= \Assume \\
          \: \perp \= \Intro{\perp}~4,5 \\
          \: S \= \Elim{\perp} \\
        \]
        \: P(x_0) \to S \= \Intro{\to}~5:7 \\
        \: \exists x (P(x) \to S) \= \Intro{\exists}~8 \\
        \: \perp \= \Intro{\perp}~2,8 \\
      \]
      \: \lnot \lnot P(x_0) \= \Intro{\lnot}~4:10 \\
      \: P(x_0) \= \Elim{\lnot\lnot} \\
    \]
    \: \all x (P(x)) \= \Intro{\all}~3:12~(\varphi := P(x_0)) \\
    \[
      \: P(x_0) \= \Elim{\all}~13 \\
      \: S \= \Elim{\to}~1,13 \\
    \]
    \: P(x_0) \to S \= \Intro{\to}~14:15 \\
    \: \exists x (P(x) \to S) \= \Intro{\exists}~16 \\
    \: \perp \= \Intro{\perp}~2:17 \\
  \]
  \: \lnot \lnot \exists x (P(x) \to S) \= \Intro{\lnot}~2:18 \\
  \: \exists x (P(x) \to S) \= \Elim{\lnot\lnot}~19 \\
\end{proofbox}

\end{document}
