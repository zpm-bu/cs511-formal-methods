\documentclass{article}
\title{Homework 10}
\author{Zachary Moring}

\usepackage{amsmath,amssymb}
\usepackage{boxproof}
\PassOptionsToPackage{hyphens}{url}
\usepackage{hyperref}
\usepackage{tikz}
\usepackage{forest}
\usepackage{enumitem}

\newcommand{\Hide}[1]{}
\newcommand{\code}[1]{\texttt{#1}}
\newcommand{\Intro}[1]{{#1}{\textrm{i}}}
\newcommand{\Elim}[1]{{#1}{\textrm{e}}}
\newcommand{\Premise}{\textrm{premise}}
\newcommand{\Let}{\textrm{let}}
\newcommand{\Assume}{\textrm{assume}}
\newcommand{\Break}{\vspace{0.2cm}\hrule{}\vspace{0.2cm}}
\newcommand{\tIf}{\text{if }}
\newcommand{\all}{\forall}
\newcommand{\defas}{\overset{\text{def}}{=}}

\newcommand{\scrM}{\mathcal{M}}

\begin{document} \maketitle

Code for the Lean portion is here:
\url{https://github.com/zpm-bu/cs511-formal-methods/blob/assignments/lean/Homework/hw10.lean}

\vspace{2cm}

\noindent\textbf{Exercise 1.} If $\scrM$ is a relational structure, the
first-order theory of $\scrM$ is:
\[
  \text{Th}(\scrM) \defas \left\{ \varphi ~\middle|~
  \varphi \text{ is a first-order sentence s.t. } \scrM \vDash \varphi
  \right\}
\]
Is $\text{Th}(\scrM)$ deductively closed?

\Break{}

% Note: A set T is /deductively closed/ if it contains every formula φ that can
% be logically deduced from T; that is, T ⊢ φ ⊃ φ ∈ T

Because we are dealing with first-order sentences, completeness tells us that
$\scrM \vDash \varphi$ is equivalent to $\scrM \vdash \varphi$.
Thus, every statement $\varphi \in \text{Th}(\scrM)$ is also in
$\overline{\scrM}$.

The same is true \textit{mutatis mutandis} to show that every statement $\psi
\in \overline{\scrM}$ is also an element of $\text{Th}(\scrM)$.

Thus by double containment, $\text{Th}(\scrM)$ is deductively closed.

\newpage{}

\noindent\textbf{Exercise 2.1.} Write a first-order sentence $\varphi_1$ which,
in any $\Sigma\prime$ structure $\scrM$ satisfying $\Gamma$, asserts
``every vertex has at least one of the colors: blue, green, purple, or
yellow.''

\Break{}

\[
  \varphi_1 \defas \forall v.~B(v) \vee G(v) \vee P(v) \vee Y(v)
\]

\vspace{2cm}

\noindent\textbf{Exercise 2.2.} Write a first-order sentence $\varphi_2$ which
asserts ``every vertex has at most one color.''

\Break{}

This is a little brute-force, but it seems like the most straightforward way
to handle it:

\begin{align*}
  \varphi_2 \defas \forall v.~ &\lnot(B(v) \wedge G(v))~\wedge~ \lnot(B(v)
  \wedge P(v)) ~\wedge~ \lnot(B(v) \wedge Y(v)) ~\wedge~ \\
  & \lnot(G(v) \wedge P(v)) ~\wedge~ \lnot(G(v) \wedge Y(v)) ~\wedge~
  \lnot(P(v) \wedge Y(v))
\end{align*}

\vspace{2cm}

\noindent\textbf{Exercise 2.3.} Write a first-order sentence $\varphi_3$ which
asserts ``no two adjacent vertices have the same color.''

\Break{}

Again, the most straightforward rule is kind of a brute-force approach, but it
definitely \textit{works}:

\begin{align*}
  \varphi_3 \defas \forall u \forall v.~ &R(u, v) \to \lnot(B(u) \wedge B(v))
  ~\wedge~ \lnot(G(u) \wedge G(v)) ~\wedge~ \\
  &\lnot(P(u) \wedge P(v)) ~\wedge~ \lnot(Y(u) \wedge Y(v)) \\
\end{align*}

\newpage{}

\noindent\textbf{Exercise 2.4.} Show that if $\scrM$ is an infinite
planar graph then there is a $\Sigma$-structure $\scrM\prime$ which
expands $\scrM$ with four unary relations $B^{\scrM\prime}$, $G^{\scrM\prime}$,
$P^{\scrM\prime}$, $Y^{\scrM\prime}$ and which satisfies $\varphi_1 \wedge
\varphi_2 \wedge varphi_3$; that is, $\scrM\prime$ is four-colorable and thus
$\scrM$ is also four-colorable.

\Break{}

In Chapter 4, we are given the fact that every finite planar graph is
4-colorable. We will rely on that fact here.

Consider a finite subset of $\scrM\prime$, the finite graph $M$. Because
$\scrM\prime$ is planar, $M$ is also planar. So $M$ is a finite planar graph
and thus $M$ is four-colorable. Thus, $M \vDash \varphi_1 \wedge \varphi_2
\wedge \varphi_3$.

Therefore since $M$ is an arbitrary finite submodel of $\scrM\prime$, by
compactness, there exists a finite model of a four-coloring for $\scrM\prime$.

\end{document}
