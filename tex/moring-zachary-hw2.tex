\title{Homework 2}
\author{Zachary Moring}
\documentclass{article}

\usepackage{amsmath,amssymb}
\usepackage{boxproof}
\usepackage{hyperref}

\newcommand{\Hide}[1]{}             % use \Hide{bla bla} to hide ``bla bla''
\newcommand{\code}[1]{\texttt{#1}}  % use \code{...} to produce ASCII chars
\newcommand{\Intro}[1]{{#1}{\textrm{i}}}
\newcommand{\Elim}[1]{{#1}{\textrm{e}}}

\newcommand*{\Break}{\vspace{0.2cm}\hrule{}\vspace{0.2cm}}
\newcommand*{\set}[1]{\left\{ #1 \right\}}
\newcommand*{\parens}[1]{\left( #1 \right)}
\newcommand*{\bracks}[1]{\left[ #1 \right]}

\begin{document} \maketitle

I was not able to complete the code exercises this week.

\vspace{1in}

\noindent\textbf{Exercise 1.} Use structural induction to prove the result
in the slide.

\Break{}

For natural numbers, I cannot see the difference between structural induction
and regular induction. To try to modify this, we can frame it as a recurrence
relation.

Let $s_1$ be 1 and $s_n = s_{n-1} + \frac{1}{n^2}$. Then $s_n \leq
	2 - \frac{1}{n}$.

For the base case, $(s_1 = 1) \leq 2 - 1$, so the proposition holds.

Assume that the proposition is true for $s_{k-1}$ and consider $s_k$. By
definition,
\[
	s_k = s_{k-1} + \frac{1}{k^2} \leq 2 - \frac{1}{(k-1)^2} + \frac{1}{k^2}
\]
which we manipulate as:
\begin{align*}
	s_k & = 2 - \frac{k^2}{k(k-1)^2} + \frac{(k-1)^2}{k(k-1)^2} \\
	    & = 2 - \frac{k^2 + (k - 1)^2}{k(k-1)^2}
\end{align*}

Now we need to show that $s_k \leq 2 - \frac{1}{k^2}$, which we can rewrite
with the denominator as
\[
	s_k = 2 - \frac{k^2 + (k - 1)^2}{k(k-1)^2} \leq 2 - \frac{(k-1)^2}{k(k-1)^2}
\]
which, by manipulation, is true if $k^2 + (k-1)^2 > (k-1)^2$, which it is.

Thus $s_k \leq 2 - \frac{1}{k^2}$.

Therefore, by structural induction on the recursion, $s_n \leq 2 - \frac{1}{n^2}$ for
all $n \geq 1$.

Therefore, $s_n < 2$ for all $n \geq 1$.



\newpage{}



\noindent\textbf{[LCS] Exercise 1.4.15.} Use induction on $n$ to prove the theorem
\[
	\parens{\varphi_1 \wedge \parens{\varphi_2 \wedge \parens{\cdots \wedge \varphi_n} \cdots } \to \psi }
	\to \parens{\varphi_1 \to \parens{\varphi_2 \to \parens{\cdots \parens{\varphi_n \to \psi } \dots } } }
\]

\Break

For the base case, consider $n=1$ and thus $\parens{ \varphi_1 \to \psi } \to
	\parens{ \varphi_1 \to \psi }$, which is trivially true. Thus, the theorem
holds for the base case.

Now, let the inductive hypothesis $IH$ be that
\[
	(\varphi_1 \wedge (\varphi_2 \wedge (\cdots \wedge \varphi_k) \cdots) \to \chi)
	\to (\varphi_1 \to (\varphi_2 \to (\cdots (\varphi_k \to \chi) \cdots ))).
\]

Thus:
\begin{proofbox}
	\: (\varphi_1 \wedge (\cdots \wedge (\varphi_k \wedge \varphi_{k+1}) \cdots )) \to \psi \= \textrm{premise} \\
	\[
		\: (\varphi_1 \wedge (\cdots \wedge \varphi_k) \cdots ) \= \textrm{assume} \\
		\[
			\: \varphi_{k+1} \= \textrm{assume} \\
			\: (\varphi_1 \wedge (\cdots \wedge (\phi_k \wedge \phi_{k+1} ))) \= \Intro{\wedge}~2~3 \\
			\: \psi \= \Elim{\to}~4~1 \\
		\]
		\: \varphi_{k+1} \to \psi \= \Intro{\to}~3:5
	\]
	\: (\varphi_1 \wedge (\cdots \wedge \varphi_k) \cdots) \to (\varphi_{k+1} \to \psi) \= \Intro{\to}~2:6
	\: (\varphi_1 \to (\cdots (\varphi_k \to (\varphi_{k+1} \to \psi)))) \= \Elim{\to}~6~IH~(\chi=6)
\end{proofbox}

Therefore, by induction, the theorem holds.



\newpage{}



\noindent\textbf{Problem 1.} Show that any of the three rules LEM, PBC, and
$\lnot\lnot\text{E}$ are interderivable.

\Break{}

Using the proof in the book on page 25, we know that PBC can be derived from
$\lnot\lnot\text{E}$.

Next, we demonstrate that LEM can be derived from PBC. No premises are needed.

\begin{proofbox}
	\[
		\: \lnot (p \vee \lnot p) \= \textrm{assumption} \\
		\[
			\: p \= \textrm{assumption} \\
			\: p \vee \lnot p \= \Intro{\vee}_1~2 \\
			\: \perp \= \Elim{\lnot}~1~3 \\
		\]
		\: \lnot p \= \Intro{\lnot}~2:4 \\
		\: p \vee \lnot p \= \Intro{\vee}_2~5 \\
		\: \perp \= \Elim{\lnot}~1~6 \\
	\]
	\: \lnot \lnot (p \vee \lnot p) \= \text{PBC}~1:7 \\
	\: p \vee \lnot p \= \Elim{\lnot\lnot}~8 \\
\end{proofbox}

Lastly, we demonstrate that $\lnot\lnot\text{E}$ is derivable from LEM.

\begin{proofbox}
	\: \lnot \lnot p \= \textrm{premise} \\
	\: p \vee \lnot p \= \textrm{LEM}~1 \\
	\[
		\: \lnot p \= \textrm{assumption} \\
		\: \perp \= \Elim{\lnot}~1~3 \\
		\: p \= \Intro{\perp}~4 \\
	\]
	\[
		\: p \= \textrm{assumption} \\
	\]
	\: p \= \Elim{\vee}~3:6
\end{proofbox}

Therefore, LEM, PBC, and $\lnot\lnot\text{E}$ are interderivable.

\end{document}
