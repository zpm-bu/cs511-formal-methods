\documentclass{article}
\title{Homework 8}
\author{Zachary Moring}

\usepackage{amsmath,amssymb}
\usepackage{boxproof}
\usepackage{hyperref}
\usepackage{tikz}
\usepackage{forest}
\usepackage{enumitem}

\newcommand{\Hide}[1]{}
\newcommand{\code}[1]{\texttt{#1}}
\newcommand{\Intro}[1]{{#1}{\textrm{i}}}
\newcommand{\Elim}[1]{{#1}{\textrm{e}}}
\newcommand{\Premise}{\textrm{premise}}
\newcommand{\Let}{\textrm{let}}
\newcommand{\Assume}{\textrm{assume}}
\newcommand{\Break}{\vspace{0.2cm}\hrule{}\vspace{0.2cm}}
\newcommand{\tIf}{\text{if }}
\newcommand{\all}{\forall}

\begin{document} \maketitle

Code for the Lean portion is here: \url{https://github.com/zpm-bu/cs511-formal-methods/blob/assignments/lean/Homework/hw8.lean}

\vspace{2cm}

\noindent\textbf{Exercise 2.2.3a.} Which of the following strings are formulae
in predicate logic? Specify a reason for failure for strings which aren't;
draw parse trees for all strings which are.

\Break{}

\noindent\textbf{i.} Yes.

\begin{center}
  \begin{forest}
    [$S$ [$m$ ] [$x$ ] ]
  \end{forest}
\end{center}

\noindent\textbf{ii.} Yes.

\begin{center}
  \begin{forest}
    [$B$ [$m$ ] [$f$ [$m$ ] ] ]
  \end{forest}
\end{center}

\noindent\textbf{iii.} $f(m)$ is not a predicate; it is a function applied to
a constant, also producing a constant.

\noindent\textbf{iv.} $B(B(m, x), y)$ is not a formula. Our definition in the
book for a formula requires that a predicate apply to \textit{terms}, and
$B(m, x)$ is not a \textit{term}, it is a predicate.

I don't actually UNDERSTAND the difference very well, but the definition 2.3.
is pretty clear about this restriction. No nesting predicates.

\noindent\textbf{v.} $S(B(m), z)$ is not a formula. $S$ and $B$ are both
predicates and we cannot nest predicates inside each other.

\newpage{}

\noindent\textbf{vi.} Yes.

\begin{center}
  \begin{forest}
    [$\to$
      [$B$ [$x$ ] [$y$ ] ]
      [$\exists z$ [$S$ [$z$ ] [$y$ ] ] ]
    ]
  \end{forest}
\end{center}


\noindent\textbf{vii.} Yes.

\begin{center}
  \begin{forest}
    [$\to$
      [$S$ [$x$ ] [$y$ ] ]
      [$S$ [$y$ ] [$f$ [$f$ [$x$ ] ] ] ]
    ]
  \end{forest}
\end{center}

\noindent\textbf{viii.} No. You cannot nest predicates.

\newpage{}

\noindent\textbf{Exercise 2.2.3b.} Let $c$ and $d$ be constants, $f$ a function
symbol with one argument, $g$ a function with two arguments, and $h$ a function
symbol with three arguments. Also let $P$ and $Q$ be predicate symbols with
three arguments. Which of the following are formulae? Draw a parse tree if so.
If not, describe why.

\Break{}

\noindent\textbf{i.} No. There is an unbound and undefined $y$ as the third
argument for $h$.

\noindent\textbf{ii.} No. As above, there is an unbound and undefined $y$.

\noindent\textbf{iii.} Yes.

\begin{center}
  \begin{forest}
    [$\forall x$
      [$Q$
        [$g$
          [$h$ [$x$ ] [$f$ [$d$ ] ] [$x$ ] ]
          [$g$ [$x$ ] [$x$ ] ]
        ]
        [$h$ [$x$ ] [$x$ ] [$x$ ] ]
        [$c$ ]
      ]
    ]
  \end{forest}
\end{center}

\noindent\textbf{iv.} No. $P$ requires three arguments, so the statement is not
well-formed.

\noindent\textbf{v.} No, I don't think so. The left hand of the implication is
$g(x, y)$, which is not a predicate -- it is a formula evaluation. Thus, there
is not a way to assign a 'truth value' to the left hand side and the
implication is not well-formed.

\noindent\textbf{vi.} Yes.

\begin{center}
  \begin{forest}
    [$Q$ [$c$ ] [$d$ ] [$c$ ] ]
  \end{forest}
\end{center}

\newpage{}

\noindent\textbf{Exercise 2.3.2} Recall that we use $\approx$ to express the
equality of elements in our models. Consider the formula
\[
  \exists x \exists y \left( \lnot (x \approx y) \wedge (\forall z ((z \approx x) \vee (z \approx y))) \right)
\]

In plain English, what does this formula specify?

\Break{}

In this universe, there are two elements $x$ and $y$ which are not related to
each other. Every other element is related to $x$ or to $y$. Thus, the space
is partitioned into two parts: The $x$ part and the $y$ part. Under $\approx$,
we can say that the set only has two elements.

\vspace{1in}

\noindent\textbf{Exercise 2.3.3.} Try to write down a sentence of predicate
logic which intuitively holds in a model if and only if the model has
(respectively):

\begin{enumerate}[label=(\alph*)]
  \item Exactly three distinct elements.
  \item At most three distinct elements.
  \item At least three distinct elements.
\end{enumerate}

\Break{}

\noindent\textbf{(a)} Consider
\begin{align*}
  \exists x \exists y \exists z (
      &\lnot (x \approx y) \wedge \lnot (x \approx z) \wedge \lnot (y \approx z) \\
      &\wedge \forall w ( (w \approx x) \vee (w \approx y) \vee (w \approx z))) \\
\end{align*}

\noindent\textbf{(b)} Here's what I thought of: There are \textit{not} four
elements.

\begin{align*}
  \lnot \exists w \exists x \exists y \exists z (
    &\lnot (w \approx x) \\
    &\lnot (w \approx y) \\
    &\lnot (w \approx z) \\
    &\lnot (x \approx y) \\
    &\lnot (x \approx z) \\
    &\lnot (y \approx z) ) \\
\end{align*}

\noindent\textbf{(c)} At \textit{least} three elements should be the same as
part (a), but without the universal binding.

\[
  \exists x \exists y \exists z (\lnot (x \approx y) \wedge \lnot (x \approx z)
  \wedge \lnot (y \approx z))
\]

\end{document}
