\documentclass{article}
\title{Homework 12}
\author{Zachary Moring}

\usepackage{amsmath,amssymb}
\usepackage{boxproof}
\PassOptionsToPackage{hyphens}{url}
\usepackage{hyperref}
\usepackage{tikz}
\usepackage{forest}
\usepackage{enumitem}

\newcommand{\Hide}[1]{}
\newcommand{\code}[1]{\texttt{#1}}
\newcommand{\Intro}[1]{{#1}{\textrm{i}}}
\newcommand{\Elim}[1]{{#1}{\textrm{e}}}
\newcommand{\Premise}{\textrm{premise}}
\newcommand{\Let}{\textrm{let}}
\newcommand{\Assume}{\textrm{assume}}
\newcommand{\Break}{\vspace{0.2cm}\hrule{}\vspace{0.2cm}}
\newcommand{\tIf}{\text{if }}
\newcommand{\all}{\forall}
\newcommand{\defas}{\overset{\text{def}}{=}}

\begin{document} \maketitle

Code for the Lean portion is here:
\url{https://github.com/zpm-bu/cs511-formal-methods/blob/assignments/lean/Homework/hw12.lean}

\vspace{2cm}

\noindent\textbf{Exercise 1.} Define $X \sim Y$ in second-order logic using a
unary function $F : X → Y$ which is both injective and surjective.

\Break{}

\begin{align*}
  X \sim Y \longleftrightarrow &\exists F (\forall x \in X (F(x) \in Y) \\
  &\wedge \forall x_1, x_2 \in X (F(x_1) \approx F(x_2) \to x_1 \approx x_2)\\
  &\wedge \forall y \in Y (\exists x \in X (F(x) \approx y)))
\end{align*}

\newpage{}

\noindent\textbf{Exercise 2.} A set $Y$ is ``countably infinite'' if $Y$ is
infinite and for every infinite subset $X$ of $Y$, there is a bijection from
$X$ to $Y$.

\textbf{(a)} Define a second-order sentence $\Psi_{\text{countably
inft}}$ such that $\mathcal{A} \vDash \Psi_{\text{countably inft}}$ if and only
if $\mathcal{A}$ is countably infinite.

\textbf{(b)} Define a second-order sentence $\Psi_{\text{uncountably
inft}}$ such that $\mathcal{A} \vDash \Psi_{\text{uncountably inft}}$ if and
only if $\mathcal{A}$ is uncountably infinite.

\Break{}

Consider
\begin{align*}
  \Psi_{\text{countably inft}}(Y) \defas &\Psi_{\text{infinite}}(Y) \\
  &\wedge \forall S \subseteq Y (\Psi_\text{infinite}(S) \to \\
  &\exists f : S \to Y (\forall s_1, s_2 (f(s_1) \approx f(s_2) \to s_1 \approx s_2) \\
  &\forall y \in Y (\exists s \in S (f(s) \approx y))))\\
\end{align*}
and
\[
  \Psi_\text{uncountably inft}(Y) \defas \Psi_\text{infinite}(Y) \wedge \lnot
  \Psi_\text{countably inft}(Y)
\]

Since countable infinity is the ``smallest'' infinite cardinality, any infinite
set which is not countable is, by definition, uncountably infinite.

\end{document}
