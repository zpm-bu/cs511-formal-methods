\documentclass{article}
\title{Homework 9}
\author{Zachary Moring}

\usepackage{amsmath,amssymb}
\usepackage{boxproof}
\PassOptionsToPackage{hyphens}{url}
\usepackage{hyperref}
\usepackage{tikz}
\usepackage{forest}
\usepackage{enumitem}

\newcommand{\Hide}[1]{}
\newcommand{\code}[1]{\texttt{#1}}
\newcommand{\Intro}[1]{{#1}{\textrm{i}}}
\newcommand{\Elim}[1]{{#1}{\textrm{e}}}
\newcommand{\Premise}{\textrm{premise}}
\newcommand{\Let}{\textrm{let}}
\newcommand{\Assume}{\textrm{assume}}
\newcommand{\Break}{\vspace{0.2cm}\hrule{}\vspace{0.2cm}}
\newcommand{\tIf}{\text{if }}
\newcommand{\all}{\forall}

\begin{document} \maketitle

Code for the Lean portion is here:
\url{https://github.com/zpm-bu/cs511-formal-methods/blob/assignments/lean/Homework/hw9.lean}

\vspace{2cm}

\noindent\textbf{Exercise 1.} The notion of \textit{two-colorable simple graphs}
coincides with the notion of \textit{bipartite simple graphs}. Write an infinite
set $\Gamma_\text{bip}$ of first-order sentences such that, for every simple
graph $G$, it holds that $G \vDash \Gamma_\text{bip}$ if and only if $G$ is
bipartite.

\Break{}

Thanks to the hint, we know that a graph $G$ is bipartite if and only if
every cycle in $G$ is even. An equivalent statement is that \textit{no} cycle
in $G$ is odd, which will be easier for us to model in sentences.

This allows us to build a simple set $\Gamma_\text{bip}$ which encodes this
statement. Let $\varphi_n$ be the sentence "$G$ has a cycle of length $n$",
which then gives us

\[
  \Gamma_\text{bip} \overset{def}{=} \left\{ \lnot \varphi_3, \lnot \varphi_5,
  \lnot \varphi_7, \dots \right\}
\]

\newpage{}

\noindent\textbf{Exercise 2.} Consider the three sentences
\begin{align*}
  \varphi_1 &\overset{\text{def}}{=} \forall x P(x, x)\\
  \varphi_2 &\overset{\text{def}}{=} \forall x \forall y (P(x, y) \to P(y, x))\\
  \varphi_3 &\overset{\text{def}}{=} \forall x \forall y \forall z
             ((P(x, y) \wedge P(y, z)) \to P(x, z))
\end{align*}
which express that the binary predicate $P$ is reflexive, symmetric, and
transitive, respectively. Show that none of these sentences is semantically
entailed by the other ones by choosing for each pair of sentences a model
which satisfies that pair, but not the third. (That is, find three binary
relations, each satisfying only two of the three properties).

\Break{}

\textbf{(Reflexive and symmetric but not transitive.)} Consider the real
numbers $\mathbb{R}$ and the simple distance metric $d(x, y) = |x - y|$. Say
that two points are related if $d(x, y) < 1$. Let $x, y, z \in \mathbb{R}$.

\begin{itemize}
  \item The relationship is \textit{reflexive} since each point is distance 0
    to itself.
  \item The relationship is \textit{symmetric} since the distance is absolute.
  \item The relationship is \textit{not transitive}. Let $x < y$ and $d(x,y)
    > 0.5$, and $y < z$ and $d(y, z) > 0.5$. Then $x$ and $z$ are on opposite
    ``sides'' of $y$ so their distance is $d(x, y) + d(y, z) > 1$ and thus
    $x$ and $z$ are not related.
\end{itemize}

\vspace{0.5cm}

\textbf{(Reflexive and transitive but not symmetric.)} Consider the simple
relationship $\leq$ on the integers.

\begin{itemize}
  \item The relationship is \textit{reflexive}, since $n \leq n$ for all $n \in
    \mathbb{Z}$.
  \item The relationship is \textit{transitive}: $x \leq y$ and $y \leq z$,
    then $x \leq z$.
  \item The relationship is \textit{not symmetric}. Note $2 \leq 3$ but $3
    > 2$.
\end{itemize}

\vspace{0.5cm}

\textbf{(Symmetric and transitive but not reflexive.)} Consider the non-negative
integers $\mathbb{N}_{+0}$. Define the relationship $\sim$ as the "nonzero
product" relationship: $n \sim m$ if $nm > 0$.

\begin{itemize}
  \item The relationship is \textit{symmetric}, since integer multiplication is
    symmetric.
  \item The relationship is \textit{transitive}. If $n \sim m$ then $n, m > 0$
    and likewise if $m \sim \ell$ then $m, \ell > 0$. Thus $n\ell > 0$ and so
    $n \sim \ell$.
  \item The relationship is \textit{not reflexive}. $0 \not\sim 0$. (In fact,
    $0 \not\sim n$ for all $n$, but this is the specific sub-behavior that is
    useful here.)
\end{itemize}
\end{document}
