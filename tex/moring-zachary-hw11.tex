\documentclass{article}
\title{Homework 11}
\author{Zachary Moring}

\usepackage{amsmath,amssymb}
\usepackage{boxproof}
\PassOptionsToPackage{hyphens}{url}
\usepackage{hyperref}
\usepackage{tikz}
\usepackage{forest}
\usepackage{enumitem}

\newcommand{\Hide}[1]{}
\newcommand{\code}[1]{\texttt{#1}}
\newcommand{\Intro}[1]{{#1}{\textrm{i}}}
\newcommand{\Elim}[1]{{#1}{\textrm{e}}}
\newcommand{\Premise}{\textrm{premise}}
\newcommand{\Let}{\textrm{let}}
\newcommand{\Assume}{\textrm{assume}}
\newcommand{\Break}{\vspace{0.2cm}\hrule{}\vspace{0.2cm}}
\newcommand{\tIf}{\text{if }}
\newcommand{\all}{\forall}
\newcommand{\defas}{\overset{\text{def}}{=}}

\begin{document} \maketitle

Code for the Lean portion is here:
\url{https://github.com/zpm-bu/cs511-formal-methods/blob/assignments/lean/Homework/hw12.lean}

\vspace{2cm}

\noindent\textbf{Exercise 2.4.12(a)} Find a model which does not satisfy
the following formula, or prove that it is valid:
\[
  (\forall x \forall y (S(x, y) \to S(y, x))) \to (\forall x \lnot S(x, x))
\]

\Break{}

Any equivalence relation will be a model which does not satisfy the formula.
In plain language, the formula reads "If $S$ is symmetric then $S$ is not
reflexive," yet \textit{every} equivalence relation has both of these
properties.

\newpage{}

\noindent\textbf{Exercise 2.4.12(b)} Find a model which does not satisfy the
following formula, or prove that it is valid:
\[
  \exists y (\forall x P(x)) \to P(y)
\]

\Break{}

\begin{proofbox}
  \: \exists y . \forall x . P(x) \= \Premise \\
  \[
    \: y_0 . \forall x . P(x) \= \Intro{\exists y}~1 \\
    \: P(y_0) \= \Elim{\forall x}~2 \\
    \: \forall x . P(x) \to P(y_0) \= \Intro{\to}~2:3
  \]
  \: P(y) \= \Elim{\exists y}~2:4
\end{proofbox}

\vspace{2cm}

\noindent\textbf{Exercise 2.4.12(c)} Find a model which does not satisfy the
following formula, or prove that it is valid:
\[
  (\forall x (P(x) \to \exists y Q(y))) \to (\forall xx \exists y (P(x) \to Q(y)))
\]

\Break{}

\begin{proofbox}
  \: \forall x . P(x) \to \exists y Q(y) \= \Premise \\
  \[
    \: x_0 \= \text{arbitrary} \\
    \: P(x_0) \to \exists y Q(y) \= \Elim{\forall x}~1 \\
    \[
      \: P(x_0) \= \Assume \\
      \: \exists y . Q(y) \= \Elim{\to}~3 \\
      \[
        \: y_0 . Q(y_0) \= \Intro{\exists y}~4 \\
        \: P(x_0) \to Q(y_0) \= \Intro{\to}~2:6 \\
      \]
      \: \exists y . P(x_0) \to Q(y) \= \Elim{\exists y}~7\\
    \]
  \]
  \: \forall x . \exists y . P(x) \to Q(y) \= \Intro{\forall x}~2:8 \\
\end{proofbox}

\newpage{}

\noindent\textbf{Exercise 2.4.12(d)} Find a model which does not satisfy the
following formula or prove that it is valid:
\[
  (\forall x \exists y (P(x) \to Q(y))) \to (\forall x (P(x) \to \exists y Q(y)))
\]

\Break{}

\begin{proofbox}
  \: \forall x. \exists y. P(x) \to Q(y) \= \Premise \\
  \[
    \: x_0 \= \text{arbitrary} \\
    \: \exists y. P(x_0) \to Q(y) \= \Elim{\forall x}~1 \\
    \[
      \: y_0. P(x_0) \to Q(y_0) \= \Intro{\exists y}~3 \\
      \[
        \: P(x_0) \= \Assume \\
        \: Q(y_0) \= \Elim{\to}~4,5 \\
        \: \exists y. P(x_0) \to Q(y) \= \Elim{\exists y}~6 \\
      \]
    \]
    \: P(x_0) \to \exists y. Q(y) \= \Intro{\to}~5:7 \\
  \]
  \: \forall x. P(x) \to \exists y. Q(y) \= \Intro{\forall x}~2:8 \\
\end{proofbox}

\end{document}
