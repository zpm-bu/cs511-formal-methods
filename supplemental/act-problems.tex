\documentclass{amsart}

\usepackage{amsmath,amssymb}

\newcommand{\surj}{\twoheadrightarrow}
\newcommand{\inv}{^{-1}}
\newcommand{\Break}{\vspace{0.2cm}\hrule{}\vspace{0.2cm}}

\title{Applied Category Theory: Problems}

\begin{document} \maketitle

\noindent\textbf{Exercise 1.21} A partition on a set $A$ can be understood in
terms of surjective functions out of $A$. Given a surjective function $f:
A \surj P$ where $P$ is any other set, the preimages $f\inv(p) \subseteq A$,
one for each element $p \in P$, form a partition of $A$.

Consider the following partition of $S := \left\{ 11, 12, 13, 21, 22, 23
\right\}$:
\[
  \left[ 11, 12 ~\middle|~ 13 ~\middle|~ 21 ~\middle|~ 22, 23 \right]
\]
Let $P := {a, b, c, d}$. Define $f$ such that $f$ models the partitions
above.

\Break{}

\begin{align*}
  f(11) = f(12) &= a \\
          f(13) &= b \\
          f(21) &= c \\
  f(22) = f(23) &= d \\
\end{align*}

\newpage{}

\noindent\textbf{Exercise 1.39} 

\end{document}
